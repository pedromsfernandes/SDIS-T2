%%%%%%%%%%%%%%%%%%%%%%%%%%
% USFD Academic Report Template
% Prof. Roger K. Moore
% University of Sheffield
% 30 July 2018
%%%%%%%%%%%%%%%%%%%%%%%%%%


\documentclass[11pt,oneside]{book}
\usepackage[margin=1.2in]{geometry}
\usepackage[toc,page]{appendix}
\usepackage{graphicx}
\usepackage{natbib}
\usepackage{lipsum}
\usepackage{caption}
\usepackage[portuguese]{babel}
\usepackage{setspace}
\usepackage{listings}
\usepackage{color}

\definecolor{dkgreen}{rgb}{0,0.6,0}
\definecolor{gray}{rgb}{0.5,0.5,0.5}
\definecolor{mauve}{rgb}{0.58,0,0.82}

\lstset{frame=tb,
  language=Java,
  aboveskip=3mm,
  belowskip=3mm,
  showstringspaces=false,
  columns=flexible,
  basicstyle={\small\ttfamily},
  numbers=none,
  numberstyle=\tiny\color{gray},
  keywordstyle=\color{blue},
  commentstyle=\color{dkgreen},
  stringstyle=\color{mauve},
  breaklines=true,
  breakatwhitespace=true,
  tabsize=3
}

\setcounter{chapter}{1}% Not using chapters, but they're used in the counters

\begin{document}

\begin{titlepage}

\begin{center}
{\LARGE Universidade do Porto}\\[1.5cm]
\linespread{1.0}\huge {\bfseries Peer-to-peer backup service for the Internet}\\[1.5cm]
\linespread{1}
\includegraphics[width=7.5cm]{feup.png}\\[1cm]

{\large Bruno Sousa \textbf{up201604145}@fe.up.pt}\\
{\large Catarina Figueiredo \textbf{up210606334}@fe.up.pt}\\
{\large Francisco Filipe \textbf{up201604601}@fe.up.ptt}\\
{\large Pedro Fernandes \textbf{up201603846}@fe.up.pt}\\[1cm]
{\large \emph{Unidade Curricular:} Sistemas Distribuídos}\\
{\large \emph{Docente:} Diana Guimarães}\\
{\large \emph{Turma:} MIEIC02}\\
{\large \emph{Grupo:} 8}\\[2cm]

\large \today
\end{center}

\end{titlepage}

% -------------------------------------------------------------------
% Contents, list of figures, list of tables
% -------------------------------------------------------------------

\doublespacing
\tableofcontents
\singlespacing

% -------------------------------------------------------------------
% Main sections (as required)
% -------------------------------------------------------------------

\mainmatter

\section{Introdução}
\paragraph{}
Este relatório foi desenvolvido no âmbito do segundo trabalho prático da 
Unidade Curricular de Sistemas Distribuídos. O seu objetivo é clarificar 
alguns dos aspetos principais da implementação do projeto 
\textit{"Peer-to-peer backup service for the Internet"}. A sua estrutura é a 
seguinte:
\begin{itemize}
    \item \textbf{Overview:} As instruções necessárias
    para compilar e executar corretamente o programa desenvolvido, tanto em Windows
    como em Linux. Também está presente nesta secção a descrição dos scripts 
    implementados, que ajudam na demonstração do trabalho.
    \item \textbf{Protocolos:} Descrição detalhada dos
    mecanismos e estruturas de dados utilizadas no desenvolvimento deste trabalho, 
    que permitem a execução concorrente dos diferentes protocolos. Esta descrição é
    acompanhada de alguns excertos do código fonte para ajudar a compreender a 
    implementação desenvolvida.
    \item \textbf{Design de Concorrência:} Descrição de cada uma das melhorias 
    propostas, da solução pensada e da implementação desenvolvida.
    \item \textbf{JSSE:} As instruções necessárias
    para compilar e executar corretamente o programa desenvolvido, tanto em Windows
    como em Linux. Também está presente nesta secção a descrição dos scripts 
    implementados, que ajudam na demonstração do trabalho.
    \item \textbf{Escalabilidade:} Descrição detalhada dos
    mecanismos e estruturas de dados utilizadas no desenvolvimento deste trabalho, 
    que permitem a execução concorrente dos diferentes protocolos. Esta descrição é
    acompanhada de alguns excertos do código fonte para ajudar a compreender a 
    implementação desenvolvida.
    \item \textbf{Tolerância a Falhas:} Descrição de cada uma das melhorias 
    propostas, da solução pensada e da implementação desenvolvida.
\end{itemize}     

\pagebreak

\section{Overview}

\section{Protocolos}

\section{Design da Concorrência}

\section{JSSE}
\paragraph{}
    Nesta implementação está também presente a utilização de JSSE. Este é utilizado em qualquer 
    tipo de comunicação entre os nós de forma a garantir uma troca de mensagens segura entre estes
    (desde pedidos e respostas de peers a transferências de ficheiros). Desta forma é protege-se 
    a identificação de cada peer (IP e porta), tornando a comunicação mais segura. Por este motivo, 
    não é possível indicar um protocolo específico em que se use este mecanismo pois ele está presente 
    em todos.

\paragraph{}
    Na implementação de JSSE optou-se pela utilização da classe javax.net.ssl.SSLSocket que adiciona
    uma camada de segurança à comunicação. Esta classe é instanciada sempre que é necessário o envio 
    de uma nova mensagem entre nós, ou seja:
    \begin{itemize}
        \item \textbf{Comunicação entre Peers:} através de mensagens que  
        \item \textbf{Comunicação entre TestApp e Peers:} através da realização do “handshake” e da 
        troca de mensagens que indicam o protocolo a realizar.
    \end{itemize}

\begin{lstlisting}
    SSLSocketFactory factory = (SSLSocketFactory)SSLSocketFactory.getDefault();
    SSLSocket socket = (SSLSocket) factory.createSocket(ip, port);

    DataOutputStream out = new DataOutputStream(socket.getOutputStream());
    BufferedReader in = new BufferedReader(new InputStreamReader(socket.getInputStream()));

    String message = "FINDSUCCESSOR " + requestId + " " + id + " \n";
    System.out.println("[Node " + requestId + "] " + message);
    out.writeBytes(message);

    String response = in.readLine().trim();
    socket.close();
\end{lstlisting}

\paragraph{}
    O excerto de código acima apresenta a implementação deste socket do lado do emissor 
    (aquele que envia a mensagem).  A implementação é relativamente simples e muito semelhante
    à de um socket desprovido de SSL. Apenas é necessário instanciar uma SSLFactory contendo a
    “default factory” e criar um SSLSocket com o ip e porta respetivos. De seguida é instanciada
    a stream de output (DataOutputStream) , encarregue de enviar a mensagem para outro nó. É 
    também neste momento que se instancia um BufferedReader responsável pela leitura da resposta. 
    De seguida constrói-se a mensagem a enviar pela stream de output. Na penúltima instrução, 
    bloqueia-se a thread atual, que fica assim à espera de uma resposta por parte do recetor. 
    Uma vez recebida esta resposta é fechado o socket.


\begin{lstlisting}
SSLServerSocketFactory ssf = (SSLServerSocketFactory) SSLServerSocketFactory.getDefault();
SSLServerSocket listenSocket = (SSLServerSocket) ssf.createServerSocket(node.port);

while (true) {
  SSLSocket connection = (SSLSocket) listenSocket.accept();
  node.executor.execute(new Runnable() {
  public void run() {
       try {
          interpretMessage(connection);
       } catch (IOException e) {
       }
        }
   });
 }

\end{lstlisting}

\paragraph{}
    Já este último pedaço de código apresenta a implementação deste mesmo socket do lado do recetor
    (aquele que recebe a mensagem). Aqui é instanciada uma SSLServerFactory onde vai estar presente 
    um SSLServerSocket.  Este último, a cada pedido que recebe, cria uma nova conexão através de um 
    SSLSocket que vai estar responsável por esse pedido. Por último é feita uma chamada à função 
    interpretMessage onde, de acordo com o pedido executado, realiza as operações necessárias e envia
    uma resposta ao emissor através de uma DataOutputStream.

\paragraph{}
    Em relação às funcionalidades do JSSE que foram utilizadas, 


\section{Escalabilidade}
\paragraph{}
    Para assegurar que o sistema desenvolvido é escalável foi implementado o Chord,
    um protocolo escalável para pesquisas num sistema peer-to-peer dinâmico (com 
    entradas e saídas frequentes de nós e com número altamente variável de nós).
    Este fornece suporte apenas para uma operação: dada uma chave, mapeia essa 
    chave num nó.  Esse nó pode ser responsável por armazenar um valor associado
    à chave. 

\paragraph{}
A utilização do Chord como protocolo torna o sistema escalável pois o custo de
uma pesquisa cresce logaritmicamente com o aumento do número de nós (log(N) em
que N é o número de nós). Como tal, até mesmo sistemas muito grandes são 
alcançáveis.

\subsection{Implementação do Chord}


\section{Tolerância a Falhas}

\paragraph{}


\end{document}