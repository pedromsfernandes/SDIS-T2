%%%%%%%%%%%%%%%%%%%%%%%%%%
% USFD Academic Report Template
% Prof. Roger K. Moore
% University of Sheffield
% 30 July 2018
%%%%%%%%%%%%%%%%%%%%%%%%%%


\documentclass[11pt,oneside]{book}
\usepackage[margin=1.2in]{geometry}
\usepackage[toc,page]{appendix}
\usepackage{graphicx}
\usepackage{natbib}
\usepackage{lipsum}
\usepackage{caption}
\usepackage[portuguese]{babel}
\usepackage{setspace}
\usepackage{listings}
\usepackage{color}

\definecolor{dkgreen}{rgb}{0,0.6,0}
\definecolor{gray}{rgb}{0.5,0.5,0.5}
\definecolor{mauve}{rgb}{0.58,0,0.82}

\lstset{frame=tb,
  language=Java,
  aboveskip=3mm,
  belowskip=3mm,
  showstringspaces=false,
  columns=flexible,
  basicstyle={\small\ttfamily},
  numbers=none,
  numberstyle=\tiny\color{gray},
  keywordstyle=\color{blue},
  commentstyle=\color{dkgreen},
  stringstyle=\color{mauve},
  breaklines=true,
  breakatwhitespace=true,
  tabsize=3
}

\setcounter{chapter}{1}% Not using chapters, but they're used in the counters

\begin{document}

\begin{titlepage}

\begin{center}
{\LARGE Universidade do Porto}\\[1.5cm]
\linespread{1.0}\huge {\bfseries Peer-to-peer backup service for the Internet}\\[1.5cm]
\linespread{1}
\includegraphics[width=7.5cm]{feup.png}\\[1cm]

{\large Bruno Sousa \textbf{up201604145}@fe.up.pt}\\
{\large Catarina Figueiredo \textbf{up210606334}@fe.up.pt}\\
{\large Francisco Filipe \textbf{up201604601}@fe.up.ptt}\\
{\large Pedro Fernandes \textbf{up201603846}@fe.up.pt}\\[1cm]
{\large \emph{Unidade Curricular:} Sistemas Distribuídos}\\
{\large \emph{Docente:} Diana Guimarães}\\
{\large \emph{Turma:} MIEIC02}\\
{\large \emph{Grupo:} 8}\\[2cm]

\large \today
\end{center}

\end{titlepage}

% -------------------------------------------------------------------
% Contents, list of figures, list of tables
% -------------------------------------------------------------------

\doublespacing
\tableofcontents
\singlespacing

% -------------------------------------------------------------------
% Main sections (as required)
% -------------------------------------------------------------------

\mainmatter

\section{Introdução}
\paragraph{}
Este relatório foi desenvolvido no âmbito do segundo trabalho prático da 
Unidade Curricular de Sistemas Distribuídos. O seu objetivo é clarificar 
alguns dos aspetos principais da implementação do projeto 
\textit{"Peer-to-peer backup service for the Internet"}. A sua estrutura é a 
seguinte:
\begin{itemize}

    \item \textbf{Overview:} Descrição das principais funcionalidades do projeto, 
    incluindo os protocolos suportados pelo serviço de backup, bem como
    um resumo das funcionalidades extra implementadas(threads, JSSE , Chord 
    e tolerância a falhas).

    \item \textbf{Protocolos:} Esta secção descreve, com recurso a excertos de código, a 
    implementação de todos os protocolos desenvolvidos de uma forma clara e concisa.

    \item \textbf{Design de Concorrência:} Aqui estão presentes os mecanismos utilizados
    para o suporte de pedidos e acessos concorrentes durante a execução do sistema.
    Mais uma vez são utilizados alguns pedaços de código para mais facilmente expôr o
    raciocínio por detrás da implementação.

    \item \textbf{JSSE:} Descrição da implementação de JSEE neste projeto. Nesta secção
    expõe-se os recursos utilizados neste projeto que adicionam uma camada de segurança
    ao mesmo. É também discutido em que situações estes são utilizados e explorada
    (com recurso a código desenvolvido) a implementação em si.

    \item \textbf{Escalabilidade:} Nesta secção está presente a implementação
    do protocolo Chord. É apresentada, de forma breve, qual a funcionalidade deste,
    sendo posteriormente abordados alguns dos conceitos principais e implementações 
    desenvolvidas no projeto.
    \item \textbf{Tolerância a Falhas:} Por último é abordada a forma como se
    previnem falhas que possam surgir ao longo da execução. Nesta secção estão presentes não 
    só a solução pensada como a sua execução.

\end{itemize}     

\pagebreak

\section{Overview}
\paragraph{}
O presente projeto tem como objetivo a implementação de um serviço de backup. Este serviço 
executa operações de Backup, Delete, Restore e Reclaim. Foi decidido utilizar JSSE para garantir 
um canal de comunicação seguro, Thread pools para a comunicação entre nós da rede, Java NIO para a 
leitura dos ficheiros e, por último, decidiu-se implementar o Chord de forma a assegurar que 
o sistema desenvolvido fosse escalável.

\subsection{Compilação}
\paragraph{}
Para compilar este projeto é necessário executar o comando seguinte no diretório onde se
encontra o projeto.
\begin{lstlisting}
    javac *.java
\end{lstlisting}

\subsection{Execução}
\paragraph{}
Para facilitar a execução do projeto foram desenvolvidos scripts para a criação dos nós e 
execução dos protocolos. Caso o sistema operativo seja windows é necessário a utilização de 
git bash para correr os scripts. 

\paragraph{}
Para executar o projeto é necessário em primeiro lugar, inicializar o   \textit{chord ring}. 
Para isso deve-se correr a seguinte instrução:

\begin{lstlisting}
java  "-Djavax.net.ssl.keyStore=keystore" 
      "-Djavax.net.ssl.keyStorePassword=sdis1822" 
      "-Djavax.net.ssl.trustStore=truststore" 
      "-Djavax.net.ssl.trustStorePassword=sdis1822" 
       Node $NODE_IP $NODE_PORT

//Script
./ring_node.sh  $NODE_IP $NODE_PORT

\end{lstlisting}

\paragraph{}
Os argumentos são respetivamente, o IP e a Porta do nó que pretendemos adicionar ao 
\textit{chord ring}. 
Caso pretendamos adicionar mais nós ao \textit{chord ring} devemos executar a seguinte
instrução:
\begin{lstlisting}
    java "-Djavax.net.ssl.keyStore=keystore" 
         "-Djavax.net.ssl.keyStorePassword=sdis1822" 
         "-Djavax.net.ssl.trustStore=truststore" 
         "-Djavax.net.ssl.trustStorePassword=sdis1822" 
         Node $NODE_IP $NODE_PORT $RING_NODE_IP $RING_NODE_PORT
    
    //Script
    ./join_node.sh $NODE_IP $NODE_PORT $RING_NODE_IP $RING_NODE_PORT
\end{lstlisting}

\paragraph{}
Os argumentos \$NODE\_IP e \$NODE\_PORT  são respetivamente, o IP e a Porta do nó que pretendemos
adicionar. Os dois últimos argumentos, \$RING\_NODE\_IP e \$RING\_NODE\_PORT são o IP e a Porta do
nó ao qual nos queremos conectar, este nó tem que pertencer ao \textit{chord ring}. 
Decidimos passar como argumento o IP do nó ao qual nos pretendemos conectar visto que, como os
computadores têm conexões de vários tipos(wifi, cabo, etc), não encontramos uma maneira de 
através de código o fazer, por isso, decidimos fazê-lo manualmente.

\paragraph{}
Só depois de iniciarmos os servidores podemos executar os protocolos desenvolvidos. Para invocar
qualquer um dos protocolos implementados é necessário correr um comando com o seguinte formato. 

\begin{lstlisting}
    java "-Djavax.net.ssl.trustStore=truststore" 
         "-Djavax.net.ssl.trustStorePassword=sdis1822" 
         TestApp $NODE_IP $NODE_PORT $PROTOCOL $OPND_1 $OPND_2
    
    //Script: 
    ./test_app.sh $NODE_IP $NODE_PORT $PROTOCOL $OPND_1 $OPND_2
\end{lstlisting}

\paragraph{}
Os dois primeiros argumentos, \$NODE\_IP e \$NODE\_PORT são o IP e a Porta do nó ao qual 
nos pretendemos conectar. O terceiro argumento, \$PROTOCOL  é o nome do protocolo que pretendemos
executar (BACKUP, DELETE, RECLAIM OU RESTORE). Os argumentos seguintes \$OPND\_1 e \$OPND\_2 
variam de protocolo para protocolo, apenas o protocolo BACKUP utiliza o \$OPND\_2.

\subsubsection{Backup}
Para invocar o protocolo Backup os parâmetros deverão ser:
\begin{itemize}
    \item \$PROTOCOL = “BACKUP”
    \item \$OPND\_1 = Nome do ficheiro
    \item \$OPND\_2 = Desired replication degree
\end{itemize}

\subsubsection{Delete}
Depois de realizado o backup podemos eliminar o backup do ficheiro através do comando Delete.
\begin{itemize}
    \item \$PROTOCOL = “DELETE”
    \item \$OPND\_1 = Nome do ficheiro
\end{itemize}

\pagebreak

\subsubsection{Restore}
Caso pretendamos recuperar o ficheiro ao qual fizemos backup podemos fazer restore do mesmo.
\begin{itemize}
    \item \$PROTOCOL = “RESTORE”
    \item \$OPND\_1 = Nome do ficheiro
\end{itemize}

\subsubsection{Reclaim}
Se pretendermos definir um espaço máximo em disco para o nó ao qual nos queremos conectar podemos executar o comando de Reclaim.
\begin{itemize}
    \item \$PROTOCOL = “RECLAIM”
    \item \$OPND\_1 = Espaço maximo em KBytes
\end{itemize}


\pagebreak

\section{Protocolos}
\paragraph{}

A comunicação entre os nós é feita através da utilização do protocolo TCP com JSSE, desta forma
garante-se uma comunicação segura e eficaz. 

\subsection{Backup}

\begin{lstlisting}
    for (int i = 0; i < chunks; i++) {
     int length = in.readInt();
     byte[] message = new byte[length];
     in.readFully(message, 0, message.length); // read the message

     String[] header = Utils.getHeader(message);
     byte[] content = Utils.getChunkContent(message, length);

     if (i == 0) {
       String key = header[1];
       BigInteger encrypted = Utils.getSHA1(key);

       ExternalNode successor = this.findSuccessor(this.id, encrypted);

       if (successor.id.equals(this.id)) {
         this.storeKey(this.id, encrypted, chunks + ":" + header[3]);
       } else {
         successor.storeKey(this.id, encrypted, chunks + ":" + header[3]);
       }
     }

     for (int j = 0; j < Integer.parseInt(header[3]); j++) {
       String key = header[1] + "-" + header[2] + "-" + j;
       BigInteger encrypted = Utils.getSHA1(key);

       ExternalNode successor = this.findSuccessor(this.id, encrypted);

       if (successor.id.equals(this.id)) {
         storeChunk(encrypted, key, content);
       } else {
         executor.execute(new ChunkSenderThread(this, successor, encrypted, key, content, false));
       }
     }
   }

\end{lstlisting}

\paragraph{}

\begin{enumerate}
    
    \item Inicialmente é executado um comando no terminal contendo: IP e a porta do nó com 
    o qual se vai comunicar; protocolo; nome do ficheiro;\textit{replication degree}. 
    Este pedido é processado pela TestApp sendo estabelecida uma comunicação com o nó 
    identificado pelo IP e porta inseridos.
    
    \item Esta, inicialmente envia o protocolo e o número de \textit{chunks} do ficheiro
    ao nó com o qual está a comunicar. De seguida abre o ficheiro correspondente e envia 
    os seus  \textit{chunks} para o nó com o qual estabeleceu a comunicação. A mensagem com os 
    \textit{chunks} tem o seguinte formato:
    \begin{lstlisting}
        BACKUP <filename> <chunkNo> <replicationDegree> <CRLF><CRLF> <chunkContent>
    \end{lstlisting}
    Termina assim, após o envio de todos os \textit{chunks}, a execução da TestApp.
    
    \item Este nó, que inicialmente já se encontra à escuta de pedidos (através da 
    \textit{thread ClientRequestListenerThread}), ao fazer a leitura do protocolo, faz o
    tratamento desse pedido executando uma chamada à função respetiva (neste caso a
    função\textit{backup}).

    \item Nesta função o nó lê o valor do número de \textit{chunks} que irá receber 
    executando posteriormente um ciclo para o tratamento dos chunks. Na primeira iteração
    é também necessário guardar informação sobre o ficheiro em si, sendo guardada uma chave
    encriptada (gerada a partir do nome do ficheiro) associada ao número de \textit{chunks} e ao
    \textit{replication degree}.
            
    \item Em cada iteração deste último ciclo, cada \textit{chunk} é guardado tantas vezes quanto
    o \textit{replication degree} previamente estabelecido. Para isso executa-se um novo ciclo
    que a cada iteração gera uma chave a partir da concatenação do número de 
    \textit{chunks} e o \textit{replication degree} atual. A partir desta chave e da 
    chamada à função \textit{findSuccessor}, descobre-se o nó onde se deve guardar esse 
    \textit{chunk}.

    \item De seguida, cria-se uma \textit{thread ChunkSenderThread} responsável pelo envio desta chave e
    da informação ao sucessor anteriormente descoberto.

    \item O nó a receber esta informação encontra-se já a executar uma outra \textit{thread}
    de tratamento de pedidos entre nós. Desta forma, ao receber o pedido de BACKUP do 
    \textit{chunk} atual, cria uma última \textit{thread ChunkReceiver Thread} que guardada
    a informação recebida.

    \item Este processo é executado para todos os \textit{chunk}, e em cada um destes é relizado
    tantas vezes quanto o \textit{replication degree} estabelecido.
\end{enumerate}
    
\pagebreak

\subsection{Restore}

\begin{lstlisting}
    for (int i = 0; i < numChunks; i++) {
     String key = fileName + "-" + i + "-0";
     BigInteger chunkID = Utils.getSHA1(key);
     keys.add(chunkID.toString());
     successor = this.findSuccessor(this.id, chunkID);

     if (successor.id.equals(this.id)) {
       storage.addRestoredChunk(chunkID.toString(), fileName, storage.readChunk(chunkID));
     } else {
       executor.execute(new ChunkRequestThread(this, successor, chunkID, fileName));
       executor.schedule(new CheckChunkReceiveThread(this, successor, key, keys, fileName, Integer.parseInt(args[1])),
           5, TimeUnit.SECONDS);
     }
   }
\end{lstlisting}

\begin{enumerate}
    \item Inicialmente é executado um comando no terminal contendo: IP e a porta do nó com o qual 
    se vai comunicar; protocolo; nome do ficheiro ao qual se pretende fazer restore; Assim como 
    no pedido anterior, este pedido é processado pela TestApp sendo estabelecida uma 
    comunicação com o nó identificado pelo IP e porta inseridos.

    \item Esta, envia o protocolo e o nome do ficheiro ao nó com o qual está a comunicar e fica 
    à espera de receber as chunks para mais tarde restaurar o ficheiro.

    \item Este nó, que inicialmente já se encontra à escuta de pedidos na \textit{thread 
    ClientRequestListenerThread} ao fazer a leitura do protocolo, faz o tratamento desse 
    pedido executando uma chamada à função respetiva (neste caso a função restore).

    \item Nesta função o nó lê o nome do ficheiro a que se pretende fazer restore e guarda-o 
    numa chave encriptada (gerada a partir do nome do ficheiro). A partir da chave e da chamada 
    à função \textit{findSuccessor} descobre-se um dos nós que fez backup desse ficheiro. 
    A partir deste nó descobre-se o número de chunks desse ficheiro. 

    \item De seguida é executado um ciclo que percorre o número total de chunks. Para cada 
    \textit{chunk} é criada uma chave encriptada (gerada a partir do número do \textit{chunk}
     , do nome do ficheiro e do \textit{replication degree}). 
     
    \item Com a utilização desta chave e da chamada à função \textit{findSuccessor}, descobre-se 
    o nó onde se deve ir buscar o \textit{chunk}. Se for o próprio, invoca-se uma função que lê 
    o \textit{chunk} do disco e o guarda em memória temporária. Caso contrário, abre-se uma 
    ligação com o sucessor e pede-se que ele envie o \textit{chunk}, guardando-0 também em memória
    temporária.

    \item Na eventualidade de uma possível falha do nó em questão, é pedido o mesmo \textit{chunk}
    com um novo \textit{replication degree} recorrendo para isso à \textit{thread CheckChunkReceiveThread}.

    \item Este processo é repetido para todos os \textit{chunks} do ficheiro. O nó que recebe os
    \textit{chunks}, após o envio de todos os pedidos, bloqueia a execução da função restore até
    receber os \textit{chunks} todos.

    \item No final é enviado para a TestApp o número total de \textit{chunks} e o conteúdo de
    cada \textit{chunk}. antes de se enviar cada \textit{chunk} é enviado o seu tamanho. 
    Esta lê toda a informação dos \textit{chunks} e restaura o ficheiro.
\end{enumerate}
 

\subsection{Delete}

\begin{lstlisting}
    for (int i = 0; i < chunks; i++) {
     for (int j = 0; j < repDegree; j++) {
       String key = fileName + "-" + i + "-" + j;
       BigInteger chunkID = Utils.getSHA1(key);
       ExternalNode chunkSuccessor = this.findSuccessor(this.id, chunkID);
       if (chunkSuccessor.id.equals(this.id)) {
         deleteChunk(this.id, chunkID);
       } else {
         chunkSuccessor.deleteChunk(this.id, chunkID);
       }
     }
   }
\end{lstlisting}

\begin{enumerate}
    \item Em primeiro lugar, é executado um comando no terminal contendo: IP e a porta do nó com 
    o qual se vai comunicar; protocolo; nome do ficheiro. Tal como anteriormente, este pedido é 
    processado mais uma vez pela TestApp, sendo estabelecida uma comunicação com o nó 
    identificado pelo IP e porta inseridos.

    \item Esta, envia o protocolo e o nome do ficheiro ao nó com o qual está a comunicar e 
    termina a sua execução.

    \item Este nó, que inicialmente já se encontra à escuta de pedidos através da
    \textit{thread ClientRequestListenerThread} ao fazer a leitura do protocolo, faz o 
    tratamento desse pedido executando uma chamada à função respetiva (neste caso a função delete).

    \item Nesta função o nó lê o nome do ficheiro e guarda-o numa chave encriptada 
    (gerada a partir do nome do ficheiro).A partir da chave e da chamada à função 
    \textit{findSuccessor} descobre-se um dos nós que fez backup do ficheiro.

    \item De seguida é executado um ciclo em que, através da função \textit{findSucessor} e de
    uma chave encriptada( gerada através do nome do ficheiro, do número do \textit{chunk} e do
    \textit{replication degree}) descobre os nós que executaram backup desse \textit{chunk} e 
    assim eliminam-se os \textit{chunks} e chaves desses \textit{chunks}. 
    
    \item Este ciclo repete-se até que todos os \textit{chunks} do ficheiro fornecido sejam 
    eliminados.
\end{enumerate}


\subsection{Reclaim}
\begin{lstlisting}
    for (File file : chunks.listFiles()) {
        String fileName = file.getName();
        BigInteger key = new BigInteger(fileName.split("\\.")[0]);
        long fileSize = file.length();
   
        if (spaceToReclaim <= 0)
          break;
        deleteChunk(this.id, key);
        spaceToReclaim -= fileSize;
      }
\end{lstlisting}

\begin{enumerate}   
    
    \item Tal como em qualquer outro protocolo, é executado um comando no terminal 
    contendo: IP e a porta do nó com o qual se vai comunicar; protocolo; tamanho máximo;  
    Assim como nos pedidos anteriores, este pedido é processado pela TestApp sendo estabelecida
    uma comunicação com o nó identificado pelo IP e porta inseridos.

    \item Esta, envia o protocolo e o espaço máximo ao nó com o qual está a comunicar e termina 
    a sua execução.

    \item Este nó, que mais uma vez já se encontra à escuta de pedidos dentro da
    \textit{thread ClientRequestListenerThread}, ao fazer a leitura do protocolo, realiza o 
    tratamento desse pedido executando uma chamada à função respetiva (neste caso a função 
    reclaim).
     
    \item Nesta função o nó lê o espaço máximo disponível e calcula o \textit{spaceToReclaim}. 
    Está variável é menor que 0 caso o espaço máximo disponível seja maior que o espaço utilizado
    pelo nó.

    \item De seguida é executado um ciclo que percorre todas as chunks que o nó guardou. Caso a 
    variável \textit{spaceToReclaim} seja maior que 0, elimina os chunks até que a variável 
    tenha uma valor menor ou igual a 0.
\end{enumerate}

 
\pagebreak

\section{Design da Concorrência}
\paragraph{}
Na implementação de mecanismos que lidassem com a concorrência de pedidos e acessos a ficheiros 
recorreu-se a:
\begin{itemize}
    \item \textit{thread pools} na comunicação entre nós.
    \item \textit{Java NIO} para a leitura dos ficheiros.
\end{itemize}

\begin{lstlisting}
    ScheduledExecutorService executor;
...
executor = Executors.newScheduledThreadPool(100);
executor.execute(new NodeThread(this));
executor.scheduleAtFixedRate(new StabilizeThread(this), 15, 15, TimeUnit.SECONDS);
\end{lstlisting}

\paragraph{}
O excerto de código acima apresenta um exemplo de implementação das \textit{threads pools} 
neste projeto. Durante a execução do programa algumas \textit{threads} são imediatamente executadas
quando são chamadas enquanto que outras (como é o caso deste exemplo) são agendadas para apenas
realizarem uma execução periódica após um certo intervalo de tempo. 

\begin{lstlisting}
    FileOutputStream out = null;
    try {
      out = new FileOutputStream(fileName);
    } catch (FileNotFoundException e) {
      // TODO Auto-generated catch block
      e.printStackTrace();
    }
    FileChannel fout = out.getChannel();
    ByteBuffer byteBuffer = ByteBuffer.allocate(chunkSize);     
\end{lstlisting}

\paragraph{}
Este último pedaço de código apresenta a utilização de Java NIO na leitura dos ficheiros. A utilização
deste mecanismo impede que um ficheiro bloqueie quando acedido por múltiplas \textit{threads} 
simultaneamente.

\pagebreak

\section{JSSE}
\paragraph{}
    Nesta implementação está também presente a utilização de JSSE. Este é utilizado em qualquer 
    tipo de comunicação entre os nós de forma a garantir uma troca segura de mensagens (desde 
    pedidos e respostas de peers a transferências de ficheiros). Desta forma protege-se a 
    identificação de cada peer (IP e porta), tornando a comunicação mais segura. Por este motivo, 
    não é possível indicar um protocolo específico em que se use este mecanismo pois ele está 
    presente em todos.

\paragraph{}
    Dos recursos disponibilizados pelo JSSE optou-se pela utilização da classe javax.net.ssl.SSLSocket.
    Esta classe é instanciada sempre que é necessário o envio de uma nova mensagem entre nós, 
    ou seja:
    \begin{itemize}
        \item \textbf{Comunicação entre Peers:} troca de mensagens que indicam pedidos a realizar
        entre \textit{peers} (p.e backup de um \textit{chunk}, restore de um \textit{chunk},
        encontrar um sucessor, entre outros - presentes na função \textit{interpretMessage} da 
        classe \textit{NodeThread}).
        \item \textbf{Comunicação entre TestApp e Peers:} realização do \textit{“handshake”} e 
        troca de mensagens que indicam os protocolos a realizar.
    \end{itemize}

\begin{lstlisting}
    SSLSocketFactory factory = (SSLSocketFactory)SSLSocketFactory.getDefault();
    SSLSocket socket = (SSLSocket) factory.createSocket(ip, port);

    DataOutputStream out = new DataOutputStream(socket.getOutputStream());
    BufferedReader in = new BufferedReader(new InputStreamReader(socket.getInputStream()));

    String message = "FINDSUCCESSOR " + requestId + " " + id + " \n";
    System.out.println("[Node " + requestId + "] " + message);
    out.writeBytes(message);

    String response = in.readLine().trim();
    socket.close();
\end{lstlisting}

\paragraph{}
    O excerto de código acima apresenta a implementação deste socket do lado do emissor 
    (aquele que envia a mensagem).  A implementação é relativamente simples e muito semelhante
    à de um socket desprovido de SSL. Apenas é necessário instanciar uma \textit{SSLFactory} contendo a
    “default factory” e criar um \textit{SSLSocket} com o ip e porta respetivos. De seguida é instanciada
    a stream de output (\textit{DataOutputStream}) , encarregue de enviar a mensagem para outro nó. É 
    também neste momento que se instancia um \textit{BufferedReader} responsável pela leitura da resposta. 
    Posteriormente constrói-se a mensagem a enviar pela stream de output. Na penúltima instrução, 
    bloqueia-se a thread atual, que fica assim à espera de uma resposta por parte do recetor. 
    Uma vez recebida esta resposta é fechado o socket. É importante referir que isto é apenas
    um exemplo de uma aplicação de um \textit{SSLSocket}.

\pagebreak

\begin{lstlisting}
SSLServerSocketFactory ssf = (SSLServerSocketFactory) SSLServerSocketFactory.getDefault();
SSLServerSocket listenSocket = (SSLServerSocket) ssf.createServerSocket(node.port);

while (true) {
  SSLSocket connection = (SSLSocket) listenSocket.accept();
  node.executor.execute(new Runnable() {
  public void run() {
       try {
          interpretMessage(connection);
       } catch (IOException e) {
       }
        }
   });
 }

\end{lstlisting}

\paragraph{}
    Já este último pedaço de código apresenta a implementação deste mesmo socket do lado do recetor
    (aquele que recebe a mensagem). Aqui é instanciada uma \textit{SSLServerFactory} onde vai estar presente 
    um \textit{SSLServerSocket}.  Este último, a cada pedido que recebe, cria uma nova conexão através de um 
    \textit{SSLSocket} que vai estar responsável por esse pedido. Por último é feita uma chamada à função 
    \textit{interpretMessage} onde, de acordo com o pedido executado, realiza as operações necessárias e envia
    uma resposta ao emissor através de uma \textit{DataOutputStream}. Mais uma vez isto é apenas uma
    aplicação de \textit{SSLSocket} do lado do recetor.

\pagebreak

\section{Escalabilidade}
\paragraph{}
    Para assegurar que o sistema desenvolvido é escalável foi implementado o Chord,
    um protocolo escalável para pesquisas num sistema peer-to-peer dinâmico (com 
    entradas e saídas frequentes de nós e com número altamente variável de nós).
    Este fornece suporte apenas para uma operação: dada uma chave, mapeia essa 
    chave num nó. Esse nó pode ser responsável por armazenar um valor associado
    à chave. 

\paragraph{}
    A utilização do Chord como protocolo torna o sistema escalável pois o custo de
    uma pesquisa cresce logaritmicamente com o aumento do número de nós (log(N) em
    que N é o número de nós na rede). Como tal, até mesmo sistemas muito grandes são 
    alcançáveis.

\subsection{Implementação do Chord}
\paragraph{}
    Esta secção centra-se apenas na implementação dos conceitos principais do protocolo Chord. 
    O seu objetivo é esclarecer um pouco estes conceitos e falar da abordagem utilizada na
    implementação dos mesmos.

    \subsubsection{Chaves}
    \paragraph{}
    No protocolo Chord toda a informação guardada é encriptada. Para isso é utilizado o algoritmo
    de encriptação \textit{SHA-1}. Este algoritmo gera chaves de 160 bits. Desta forma, 
    este é o número máximo de entradas presente na \textit{finger table} de cada nó, pois 
    representa o valor mínimo de nós necessário para conseguir chegar a qualquer outro nó de
    uma rede de N nós com um custo de log(N). 
    \paragraph{}
    Existem 2 tipos de chaves: file-info e chunk. 
    \begin{itemize}
        \item \textbf{File-info:} chave contendo informação acerca do nome do ficheiro, o número
        de \textit{chunks} que esse mesmo ficheiro contém e o \textit{replciation degree} deste.
        \item \textbf{Chunk:} chave contendo informação sobre o nome do ficheiro, o número do
        \textit{chunk} atual e o valor do \textit{replication degree} atual.
    \end{itemize}

    \paragraph{}
    Os identificadores dos nós também são encriptados, estes contém os valores de IP e porta do
    nó respetivo.

    \subsubsection{Chord Ring}
    \paragraph{}
    Dentro do Chord, para que um nó consiga chegar a todos os outros nós e de forma a que estes
    cheguem a ele, é necessário guardar, para cada nó, o seu antecessor, o sucessor e uma 
    \textit{finger table} (que será abordada na secção seguinte). O conjunto de todos os nós
    de um sistem organizado desta forma é comumente designado por \textit{ring}. A implementação 
    desta caraterística no sistema desenvolvido foi realizada através da criação de 3 atributos 
    na classe \textit{Node}, sendo estes:
    \begin{itemize}
        \item \textbf{predecessor:} Nó que antecede o nó atual.
        \item \textbf{successor:} Nó que sucede o nó atual.
        \item \textbf{fingerTable:} Nós que estão ligados ao nó atual.
    \end{itemize}

    \subsubsection{Finger Table}
    \paragraph{}
    Para evitar a pesquisa linear, o Chord implementa um método de pesquisa mais rápido, exigindo
    que cada nó mantenha uma tabela contendo até m entradas, lembrando que m é o número de bits na 
    chave encriptada por \textit{SHA-1}. A primeira entrada da \textit{finger table} é, na verdade,
    o sucessor imediato do nó. Sempre que um nó pretende pesquisar uma chave, apenas necessita de
    consultar o sucessor ou predecessor mais próximo na sua \textit{finger table}, até que um nó 
    descubra que a chave está armazenada no seu sucessor imediato.

    \paragraph{}
    Com essa \textit{finger table}, o número de nós que devem ser contactados para localizar 
    um sucessor na rede de N nós é O(log N).

    \paragraph{}
    Na implementação desta \textit{finger table} apenas se criou um atributo na classe \textit{Node}
    que consiste num array de \textit{External Node}. Este atributo está constantemente a ser
    alterado pela \textit{thread StabilizeThread}, que periodicamente modifica esta estrutura de dados
    de forma a ficar o mais atual possível.

    \subsubsection{Find Successor}
    \paragraph{}
    Esta é a operação central deste protocolo. A sua função é, dada uma chave, retornar o nó onde
    esta chave deve ser guardada. Para isso implementou-se a seguinte função:
    \begin{lstlisting}
        public ExternalNode findSuccessor(BigInteger requestId, BigInteger id) {
            if (id.equals(this.id))
              return this;
        
            if (idBetween(id, this.id, this.successor.id))
              return this.successor;
        
            ExternalNode n0 = closestPrecedingNode(id);
        
            if (n0.id.equals(this.id))
              return this;
        
            return n0.findSuccessor(this.id, id);
          }
        
    \end{lstlisting}

    \paragraph{}
    Esta função está presente na classe \textit{Node} e consiste num \textit{override} à função
    da classe mãe \textit{External Node}. A sua finalidade é retornar o nó onde deverá ser guardada
    a chave \textit{id} (segundo parâmetro da função acima apresentada). Caso essa chave seja
    igual ao nó atual, então é neste que deve ser guardada. Senão, caso esta se encontre entre
    o nó atual e o seu sucessor, deverá ser guardada neste mesmo sucessor. Caso nenhuma destas condições
    seja cumprida, recorrendo à \textit{finger table} é executada uma chamada a esta mesma função
    no nó que se encontrar mais próximo da chave (\textit{id}) e que a antecede. Isto vai gerar um 
    pedido que vai ser enviado a esse nó (através da função \textit{findSuccessor} do \textit{ExternalNode}).
    Este último, ao receber tal pedido, volta o pedaço de código acima apresentado. Este processo repete-se
    até que seja encontrado o sucessor da chave em questão, ou seja, o \textit{id}.

    \pagebreak

    \subsubsection{Estabilização}
    \paragraph{}
    Para garantir pesquisas corretas, todos os sucessores devem estar atualizados. Portanto, um 
    protocolo de estabilização é executado periodicamente em segundo plano, o que atualiza 
    as \textit{finger tables} e sucessores.

    \paragraph{}
    O protocolo de estabilização funciona da seguinte maneira:
    \begin{itemize}
        \item \textbf{stabilize():} N pergunta ao seu sucessor pelo predecessor P e decide se P deve 
        ser o sucessor de N ou não (isto acontece caso P tenha entrado recentemente no sistema).
        \begin{lstlisting}
            public void stabilize() {
                ExternalNode x = this.successor.getPredecessor(this.id);

                if (x != null && !this.id.equals(x.id)
                    && (this.id.equals(this.successor.id) || idBetween(x.id, this.id, successor.id))) {
                this.successor = x;
                fingerTable[0] = x;
                }

                this.successor.notify(this.id, this);
            }
        \end{lstlisting}

        \item \textbf{fixFingers():} Realizar um pequeno \textit{update} à \textit{finger table}.
        Visto que podem haver entradas de novos nós, e saídas de outros, convém ter uma versão o
        mais atualizada possível da \textit{finger table}.
        
        \begin{lstlisting}
            private BigInteger calculateFinger(int i) {
                return ((this.id.add(new BigInteger("2").pow(i))).mod(new BigInteger("2").pow(fingerTable.length)));
            }

            public void fixFingers() {
                for (int i = 1; i < fingerTable.length; i++) {
                BigInteger fingerID = calculateFinger(i);
                fingerTable[i] = findSuccessor(this.id, fingerID);
                }
            }
        \end{lstlisting}

        \item \textbf{checkPredecessor():} Notifica o sucessor de N da sua existência para que 
        este possa alterar o seu antecessor para N. 
        \begin{lstlisting}
            public void checkPredecessor() {
                if (this.predecessor != null && this.predecessor.failed(this.id))
                  this.predecessor = null;
            }
        \end{lstlisting}

    \end{itemize}        


\pagebreak

\section{Tolerância a Falhas}

\paragraph{}
    Como forma de prevenir que saídas ou encerramentos inesperados de alguns nós do 
    sistema causem a perda da informação contida nesse nó (chaves ou chunks), implementou-se
    um protocolo de tolerância a falhas. Esse protocolo tem por base a utilização
    de \textit{replication degree}, semelhante ao já implementado no primeiro projeto desta
    Unidade Curricular.

\subsection{Implementação da Tolerância a Falhas}
    \paragraph{}
    De forma a tornar esta explicação mais clara será dado como exemplo a implementação e
    importância do \textit{replication degree} nos dois protocolos mais relevantes: \textbf{BACKUP} 
    e \textbf{RESTORE}. Os protocolos DELETE e RECLAIM não são aqui referidos pois a existência 
    ou não de um mecanismo de tolerância a falhas não afeta estes protocolos.
    \begin{itemize}
        \item \textbf{BACKUP:} Neste protocolo, caso não existisse \textit{replication degree}, 
        ao realizar-se um \textit{backup} de um \textit{chunk} apenas seria gerada uma chave com o nome 
        do ficheiro e número do \textit{chunk} atual. Esta chave, após encontrado o seu sucessor,
        seria guardada no respetivo nó. O problema desta implementação baseia-se no facto de que,
        caso este último nó inesperadamente sofra uma falha (p.e. abandonar o sistema), a informação
        desse \textit{chunk} perde-se por completo, comprometendo assim o bom funcionamento de todo o sistema.
        
        \begin{lstlisting}
            for (int j = 0; j < Integer.parseInt(header[3]); j++) {
                String key = header[1] + "-" + header[2] + "-" + j;
                BigInteger encrypted = Utils.getSHA1(key);

                ExternalNode successor = this.findSuccessor(this.id, encrypted);

                if (successor.id.equals(this.id)) {
                storeChunk(encrypted, key, content);
                } else {
                executor.execute(new ChunkSenderThread(this, successor, encrypted, key, content, false));
                }
             }
        \end{lstlisting}
        
        \paragraph{}
            Com a introdução deste \textit{replication degree} a cada \textit{chunk} são geradas 
            várias chaves (tantas quanto o valor do \textit{replication degree} - header[3] no 
            código acima) utilizando o nome  do ficheiro, o número do \textit{chunk} atual e o 
            valor do \textit{replication degree} atual.
            Apesar de, num determinado \textit{chunk}, apenas variar o \textit{replication degree},
            esta mudança já é o suficiente para que sejam geradas chaves distintas para 
            \textit{replication degrees} distintos. Desta forma, cada cópia desse \textit{chunk} 
            será aleatoriamente espalhada pelo sistema, diminuindo a probabilidade de que, ao 
            falhar um nó, haja perda imediata de um \textit{chunk}.
        
            \pagebreak

            \item \textbf{RESTORE:} Este protocolo está encarregue de pesquisar o sistema por 
            \textit{chunks} de um ficheiro para restaurar no seu nó. Caso não se tivesse em conta
            esta tolerância a falhas, no momento de BACKUP de um ficheiro cada \textit{chunk} seria 
            apenas guardado uma única vez. Desta forma, no momento do pedido RESTORE desse ficheiro,
            caso algum nó, por motivo de falha, não se encontrasse ligado ao sistema, tornar-se-ia 
            impossível restaurar o ficheiro em questão.
        
        \begin{lstlisting}
            for (int i = 0; i < numChunks; i++) {
                String key = fileName + "-" + i + "-0";
                BigInteger chunkID = Utils.getSHA1(key);
                keys.add(chunkID.toString());
                successor = this.findSuccessor(this.id, chunkID);

                if (successor.id.equals(this.id)) {
                    storage.addRestoredChunk(chunkID.toString(), fileName, storage.readChunk(chunkID));
                } else {
                    executor.execute(new ChunkRequestThread(this, successor, chunkID, fileName));
                    executor.schedule(new CheckChunkReceiveThread(this, successor, key, keys, fileName, Integer.parseInt(args[1])),
                        5, TimeUnit.SECONDS);
                }
            }
        \end{lstlisting}
        
        \paragraph{}
            Com a introdução do \textit{replication degree}, é gerada uma chave contendo, tal como 
            no protocolo BACKUP, o nome  do ficheiro, o número do \textit{chunk} atual e o valor 
            do \textit{replication degree} atual. Com esta chave e recorrendo à função
            \textit{findSuccessor} descobre-se o nó contendo este chunk. Caso este nó não seja 
            ele próprio, é criada uma \textit{thread} responsável pelo envio do pedido RESTORE 
            ao nó correspondente. Para além disso, é criada ainda uma outra \textit{thread} 
            responsável por verificar se o \textit{chunk} pedido já foi recebido ou não. Caso 
            não tenha sido recebido (por falha do nó ao qual se pediu o \textit{chunk}), é pedido
            o mesmo \textit{chunk} mas com o \textit{replication degree} imediatamente acima do 
            anterior.

        \paragraph{}
            Desta forma é possível verificar que, uma falha num nó do sistema já não compromete
            por completo o bom funcionamento do sistema visto que, com um \textit{replication degree}
            associado, a probabilidade de um ficheiro ficar incompleto no sistema reduz substancialmente.
           
            
    \end{itemize}

\end{document}